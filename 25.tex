\section{Clase 25}
En la clase pasada vimos que las fuerzas nucleares no dependen de la carga eléctrica de los nucleones. Es decir, un nucleón puede tener isospin up (protón) o isospin down (neutrón), cada uno con su respectiva función de onda $\psi_p$ y $\psi_n $. La función de onda del nucleón es
\begin{equation}
  \psi =\mqty(\psi_p\\\psi_n)
\end{equation}

No estamos considerando la interacción electromagnética. La interacción nuclear \textit{no} distingue un protón de un neutrón, luego, la interacción nuclear es invariante bajo el inetrcambio de un protón por un neutrón.


De acuerdo con el principio de superposición de la mecánica cuántica:
\begin{align}
  \psi_p'&=\a \psi_+ +\b\psi_n\\
  \psi_n'&=\g  \psi_+ +\d \psi_n
\end{align}
Entonces, ¿cómo trasforma el nucleón?,
\begin{align}
  \mqty(\psi_p'\\ \psi_n')=\mqty(\a&\b \\\g&\d )\mqty(\psi_p\\\psi_n)
\end{align}
Si llamamaos 
\begin{equation}
  U=\mqty(\a&\b \\\g&\d )
\end{equation}
entonces
\begin{equation}
\boxed{  \psi'=U\psi}
\end{equation}
Las transformaciones son invariantes bajo las transformacion $\psi'=U\psi$ donde $U$ es una matriz compleja de $2\times 2$.

¿Cómo afecta a la matriz $U$ la mecánica cuántica? Sabemos que en mecánica cuánica la probabilidad se consrerva, es decir,
\begin{equation}
  |\psi'|^2=|\psi|^2
\end{equation}
donde $\psi'=U\psi$, pero
\begin{align}
  |\psi'|^2=\bra{\psi'}\ket{\psi'}=\braket{\psi}=|\psi|^2
\end{align}
\begin{equation}
  \implies \braket{U\psi}=\braket{\psi}
\end{equation}
\begin{equation}
  \implies \bra{\psi}U^\dagger U\ket{\psi}=\braket{\psi}
\end{equation}
\begin{equation}\label{25.1}
  \implies U^\dagger U=\id 
\end{equation}
Es decir, la matriz $U$ es una matriz unitaria,
\begin{equation}
  U^\dagger=U^{-1}
\end{equation}
De \eqref{25.1} vemos que $\det(U^\dagger U)=\det(\id )$
\begin{equation}
  \implies \det U^\dagger \det U=1 \implies (\det U)^2=1
\end{equation}
\begin{equation}
  \implies \det U=+1
\end{equation}
Es decir, tenemos el grupo $SU(2)$. Tambien sabemos del álgebra lineal, que una matriz unitaria se puede exponenciar con una matriz hermítica,
\begin{equation}
  U=e^{i\a }\qquad \text{donde } \a=\a^\dagger
\end{equation}
es decir, $\a$ es hermítica. Esto implica que
\begin{equation}
  \det U=\det (e^{i\a })=1
\end{equation}
Pero el álgebra lineal nos enseña que si $A$ es una matriz, entonces $\det e^{A}=e^{\Tr A}$, lo que implica que
\begin{align}
  \det U=\det e^{i\a }=e^{\Tr (i\a )}=1
\end{align}
\begin{equation}
  \implies \boxed{\Tr \a =0}
\end{equation}

Si $\{T_a\}$ son una base para las matrices hermíticas $\a$, entonces $\a=\a^{a}T_a$,
\begin{equation}
  \implies \boxed{U=e^{i\a }=e^{i\a^{a}T_a}}
\end{equation}
Estos resultados son válidos en general para el grupo $SU(N)$.

En el caso $N=2$, las matrices bases $\{T_a\}$ vienen dadas por las matrices de Pauli. En el caso $N=3$, los $\{T_a\}$ son las matrices de Gell-Mann.

Las bases $\{T_a\}$ son bases, en general, no-coordenadas \footnote{Las bases coordenadas conmutan.}, por lo cual satisfacen las relaciones de conmutación
\begin{equation}
  [T_a,T_b]=if_{ab}^{~~c}T_c
\end{equation}

\subsection{Teoría de Yang-Mills}
Consideremos ahora el grupo $SU(N)$ cuya álgebra de Lie asociada $su(N)$ tiene como generadores a la bae $\{T_a\}$ que satisface la relación de conmitación
\begin{equation}
  [T_a,T_b]=if_{ab}^{~~c}T_c
\end{equation}
Si $U\in SU(N)$ entonces $U=e^{i\a }\equiv e^{i\a ^{a}T_a}$.

Una función de onda nuclear transforma bajo $SU(N)$ global como
\begin{align}
  \psi&\to \psi'=U\psi\\
  \bar{\psi}&\to \bar{\psi}'=\bar{\psi}U^\dagger
\end{align}
Consideremos el Lagrangeano
\begin{equation}
  \boxed{\L_0=i\bar{\psi}\g^\m \partial_\m \psi}
\end{equation}

\begin{teor}
	El Lagrangeano $\L_0$ es invariante bajo el grupo de transformaciones $SU(N)$ global
	\begin{align}
  \psi&\to \psi=U\psi\\
  \bar{\psi}&\to \bar{\psi}'=\bar{\psi}U^\dagger
\end{align}
donde $U^\dagger U=UU^\dagger=\id$, $U=e^{i\a }\equiv e^{i\a^{a}T_a}$, donde $\a^{a}=$ constante.
\end{teor}

\begin{prueba}
	\begin{align}
  \L_0'&=i\bar{\psi}'\g^\m \partial_\m \psi'\\&=i\bar{\psi}U^\dagger \g^\m \partial_\m (U\psi)\\
  &=i\bar{\psi}U^\dagger U\g^\m\partial_\m \psi\\
  &=i\bar{\psi}\g^\m \partial_\m \psi\\
  &=\L_0\qquad \checkmark 
\end{align}
\end{prueba}

Esto ocurre debido a que
\begin{equation}
  \L_0'=i\bar{\psi}'\g^\m (\partial_\m \psi)'=i(\bar{\psi}U^\dagger)(\g^\m U \partial_\m \psi)
\end{equation}
es decir, la invariancia se debe a que
\begin{equation}
  (\partial_\m \psi)'=U\partial_\m \psi\rightarrow \partial_\m '\psi'=U\partial_\m \psi
\end{equation}
\begin{equation}
  \implies \boxed{\partial_\m '=U\partial_\m U^\dagger}
\end{equation}

La idea implementada por Weyl y luego por Yang-Mills es pasar de $SU(N)$ global a $SU(N)$ local. 

$SU(N)$ local es dado por
\begin{align}
  \psi&\to \psi'=U(x)\psi\\
  \bar{\psi}&\to \bar{\psi}'=\bar{\psi }U^\dagger (x)
\end{align}
donde $U^\dagger U=UU^\dagger =\id $,
\begin{equation}
  U=e^{i\a (x)}=e^{i\a^{a}(x)T_a}
\end{equation}

Estudiemos ahora el comportamiento de $\L_0=i\bar{\psi}\g^\m \partial_\m \psi$ bajo $SU(N)$ local,
\begin{align}
  \L_0'&=i\bar{\psi}'\g^\m \partial_\m \psi'\\
  &=i\bar{\psi}U^\dagger (x )\g^\m \partial_\m (U(x)\psi)
\end{align}
pero $\partial_\m (U(x)\psi)=U(x)\partial_\m \psi+[\partial_\m U(x)]\psi$, con $\partial_\m U(x)=\partial_\m e^{i\a(x)}=e^{i\a(x)}i\partial_\m \a(x)=U(x)i\partial_\m \a (x)$. Esto implica que
\begin{align}
  \L_0'&=i\bar{\psi}U^\dagger (x )\g^\m \partial_\m (U(x)\psi)\\
  &=i\bar{\psi}U^\dagger(x )\g^\m [U(x)\partial_\m \psi+U(x)i\partial_\m \a(x)\psi]
\end{align}
\begin{equation}\label{25.2}
  \implies \boxed{\L_0'=i\bar{\psi}U^\dagger(x)\g^\m U(x)[\partial_\m +i\partial_\m \a (x)]\psi}
\end{equation}
Es decir, $\L_0$ no es invariante bajo $SU(N)$ local. El siguiente paso es restaurar la invariancia de $\L_0$.

Escribiendo
\begin{equation}\label{25.3}
  \L_0'=i\bar{\psi}'\g^\m (\partial_\m \psi)'
\end{equation}
comparando \eqref{25.2} con \eqref{25.3}
\begin{equation}
  \bar{\psi}=\bar{\psi}U^\dagger
\end{equation}
\begin{align}
  (\partial_\m \psi)'&=U(x)[\partial_\m +i\partial_\m \a(x)]\psi\\
  \partial_\m '\psi '&=U(x)[\partial_\m +i\partial_\m \a (x)]\psi
\end{align}
\begin{equation}
 \boxed{ \partial_\m '=U(x)[\partial_\m +i\partial_\m \a(x)]U^\dagger(x)}
\end{equation}
Para restaurar la invariancia es necesario definir una nueva derivada $D_\m $ que bajo $SU(N)$ local, transforme como 
\begin{equation}
  D_\m \to D_\m '=U(x)D_\m U^\dagger (x)
\end{equation}
Por lo que el Lagrangeano invariante debe venir dado por
\begin{equation}
  \L_0=i\bar{\psi}\g^\m D_\m \psi
\end{equation}
¿Cómo definimos la derivada $D_\m $? Teniendo en cuenta que
\begin{equation}\label{25.star}
  \partial_\m '=U(x)[\partial_\m +i\partial_\m \a(x)]U^\dagger (x)
\end{equation}
y que
\begin{equation}\label{25.4}
  D_\m '=U(x)D_\m U(x)^\dagger
\end{equation}
definimos $D_\m =\partial_\m +iA_\m $ donde $A_\m$ debe transformar bajo $SU(N)$ local respetando la ley de transformación \eqref{25.4}.

\begin{teor}
	Bajo una transformación $SU(N)$ local, el campo potencial de gauge $A_\m $ transforma como 
	\begin{equation}
  A_\m \to A_\m '=UA_\m U^\dagger+i[\partial_\m U]U^\dagger,\qquad U=U(x)
\end{equation}
o bien
\begin{equation}
  A_\m \to A_\m '=UA_\m U^\dagger -iU[\partial_\m U^\dagger ]
\end{equation}
\end{teor}

\begin{prueba}
	De \eqref{25.4},
	\begin{align}
  \partial_\m '+iA_\m '&=U[\partial_\m +iA_\m ]U^\dagger
\end{align}
Usando \eqref{25.star} podemos escribir 
\begin{align}
  U[\partial_\m +i\partial_\m \a ]U^\dagger +iA_\m'&=U[\partial_\m +iA_\m ]U^\dagger \\
  \cancel{U\partial_\m U^\dagger} +i(U\partial_\m \a )U^\dagger+iA_\m'&=\cancel{U\partial_\m U^\dagger} +iUA_\m U^\dagger\\
  iA_\m '&=iUA_\m U^\dagger -i(U\partial_\m \a )U^\dagger \\
  A_\m '&=UA_\m U^\dagger -(U\partial_\m \a )U^\dagger
\end{align}
pero $\partial_\m U=\partial_\m e^{i\a }i\partial_\m \a =i(U\partial_\m \a )$. Luego,
\begin{align}
 \Aboxed{ A_\m '&=UA_\m U^\dagger +i(\psi_\m U)U^\dagger }\qquad \checkmark
\end{align}
Pero, $\partial_\m (UU^\dagger)=(\partial_\m U)U^\dagger +U(\partial_\m U^\dagger)=0$, de manera que
\begin{align}
  (\partial_\m )U^\dagger =-U(\partial_\m U^\dagger)
\end{align}
\begin{equation}
 \implies \boxed{A_\m '=UA_\m U^\dagger -iU(\partial_\m U^\dagger )}\qquad \checkmark
\end{equation}
\end{prueba}

Así entonces el Lagrangeano invariante bajo $SU(N)$ local será
\begin{equation}
  \L=i\bar{\psi}\g^\m D_\m \psi=i\bar{\psi}\gamma^\m (\partial_\m +iA_\m )\psi
\end{equation}
\begin{equation}\label{25.6}
  \L=i\bar{\psi}\g^\m \partial_\m \psi-\bar{\psi}\g^\m \psi A_\m 
\end{equation}
donde $A_\m =A_\m ^{a}T_a$.

 Pero los $T_a$ son los generadores de $SU(N)$ que son $N^2-1$ generadores, lo que implica que para restaurar la invariancia es necesario introducir $N^2-1$ campos de gauge $A_\m$ (uno por cada generador).
 
 El Lagrangeano \eqref{25.6} contiene $N^2-1$ campos compensantes pero no se observan términos cinéticos. Siguiendo el método usual obtenemos el término cinético,
 \begin{equation}
  [D_\m ,D_\n ]\psi	=iF_{\m\n }\psi
\end{equation}
Bajo $SU(N)$ local,
\begin{align}
  [D_\m ',D_\n ']\psi&=iF_{\m\n }'\psi
\end{align}
pero
\begin{align}
  [D_\m ',D_\n ']&=[UD_\m U^\dagger ,UD_\n U^\dagger ]\\
  &=UD_\m U^\dagger UD_\n U^\dagger -UD_\n U^\dagger UD_\m U^\dagger \\
  &=U(D_\m D_\n -D_\n D_\m )U^\dagger \\
  &=U[D_\m ,D_\n ]U^\dagger\\
  &=U(iF_{\m\n })U^\dagger\\
  &=iF_{\m\n }'
\end{align}
\begin{equation}
  \implies \boxed{F_{\m\n }'=UF_{\m\n }U^\dagger }
\end{equation}
es decir, $F_{\m\n }$ es un invariante de gauge.

Veamos cual es su forma
\begin{align*}
  iF_{\m\n }\psi &=[D_\m ,D_\n ]\psi\\
  &=(\partial_\m +iA_\m\ )(\partial_\n \psi+i\partial_\n \psi)-(\partial_\n +i\partial_\n )(\partial_\m\psi +iA_\m \psi)\\
  &=\partial_\m \partial_\n \psi +i(\partial_\m A_\n  )\psi+iA_\n \psi_\m \psi+iA_\m \psi_\n \psi-A_\m A_\n \psi -\partial_\n \partial_\m \psi-i(\partial_\n A_\m )\psi-iA_\m \psi_\n \psi +A_\n A_\m \psi\\
  &=i(\partial_\m A_\n -\partial_\n  A_\m )\psi -(A_\m A_\n -A_\n A_\m )\psi
\end{align*}
\begin{equation}
  \implies iF_{\m\n }=i(\partial_\m A_\n -\partial_\n A_\m )-[A_\m ,A_\n ]\
\end{equation}
\begin{equation}
  \implies \boxed{F_{\m\n }=\partial_\m A_\n -\partial_\n A_\m +i[A_\m ,A_\n ]}
\end{equation}

Notemos que el conmutador no es cero debido a que
\begin{align}
  A_\m &=A_\m ^{a}T_a\\
  A_\n  &=A_\n ^bT_b\\
  A_\m A_\n &=A_\m ^{a}A_\n ^b T_aT_b\\
  A_\n A_\m &=A_\n ^bA_\m ^{a}T_bT_a\\
  [T_a,T_b]&=if_{ab}^{~~c}T_c
\end{align}






















 









































































































