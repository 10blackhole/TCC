\section{Clase 2}
\begin{enumerate}
	\item Las ecuaciones de Newton son \textit{invariantes en forma} bajo las transformaciones de Galileo.
	\item Las ecuaciones de Newton sn ecuaciones de segundo orden en $\vb*{x}_\alpha$. Es bueno recalcar que \textit{todas las ecuaciones dinámicas de la física fundamental son de segundo orden}. Las ecuaciones de orden mayor al segundo, tienden a tener inestabilidades \cite{Ostrogradsky:1850fid}.
	\item Si $L=L(q_i,\dot{q}_i,t)$ es la función de Lagrange para un sistema mecánico, entonces la dinámica del sistema es gobernada por las ecuaciones de Euler-Lagrange,
	\begin{equation}\label{eqs EL}
  [L]_i=\pdv{L}{q_i}-\dv{t}\pdv{L}{\dot{q}_i}=0
\end{equation}
Estas ecuaciones no cambian si la función de Lagrange es modificada a la forma
\begin{equation}
  \tilde{L}=L+\dv{t}B(q,t)
\end{equation}
con $L=L(q_i,\dot{q}_i,t)$ y $\bar{L}=\bar{L}(q_i,\dot{q}_i,t)$
\item La libertad en la elección de las coordenadas generalizadas implica que las ecuaciones de Euler-Lagrange son estructuralmente invariantes bajo un cambio de coordenadas:
\begin{equation}\label{2.1}
  q_i\to q'_i=q'_i(q_l)
\end{equation}
lo cual implica que
\begin{equation}
  \boxed{[L']_k=[L]_l\pdv{q_l}{q'_k}}
\end{equation}
que muestra que la derivada de Euler transforma como un vector covariante bajo la transformación \ref{2.1}
\end{enumerate}
\begin{equation}
  \mbox{Si }[L]_l=0 \Rightarrow [L']_k=0.
\end{equation}

Es importante recalcar que la invariancia estructural es distinto a la invariancia en forma (covariancia).

\textit{Todos los observadores observan la misma forma de las ecuaciones de los modelos de la naturaleza.}

\begin{ej}
	La ecuación de Newton en el SRI K toma la forma $\vb*{F}=m\vb*{a}$ mientras que en el SRI $K'$ toma la forma $\vb*{F'}=m'\vb*{a}'$.
\end{ej}
\begin{ej}
	Las ecuaciones de Maxwell tendrán la misma forma en todos los SRI.
\end{ej}

Notemos son embargo, que en la mecánica de Newton las transformaciones son las transformaciones de Galileo y que en la electrodinámica de Maxwell son las transformaciones de Lorentz.

La covariancia de las ecuaciones del movimiento bajo una transformación de coordenadas es la propiedad que define una \textbf{simetría de Lie}.

\subsection{Acerca de la matriz Hessiana}
Una característica básica de las ecuaciones de Newton $m\ddot{\vb*{r}}=\vb*{F}(\vb*{r},\dot{\vb*{r}},t)$ es que es posible expresar la aceleración $\ddot{\vb*{r}}$ en función de la posición $\vb*{r}$, de la velocidad $\dot{\vb*{r}}$ y de $t$,
\begin{equation}
  \ddot{\vb*{r}}(t)=\frac{1}{m}\vb*{F}(\vb*{r},\dot{\vb*{r}}(t))
\end{equation}
Esta es una formulación vectorial de la mecánica  es basada en el concepto d partícula material. Esto llevó a pensar que la naturaleza podría no ser contínua, sino que podría ser atómica (cuántica). Esto condujo a la formulación escalar de la mecánica representado de la introducción del concepto de energía.

La formulación de Lagrange y de Hamilton fue el resultado de esta búsqueda. Sin embargo, de las ecuaciones de Euler-Lagrange \eqref{eqs EL} no es evidente cómo expresar la aceleración $\ddot{q}(t)$ en función de $q(t), \dot{q}(t)$ y $t$.

Consideremos
\begin{equation}
	\dv{t}\pdv{L}{\dot{q}_n},\qquad L=L(q_n,\dot{q}_n,t)
\end{equation}
notemos que
\begin{equation}
  \dv{t}\pdv{L}{\dot{q}_n}=\pdv{L}{\dot{q}_n}{\dot{q}_m}\ddot{q}^m+\pdv{L}{\dot{q}_n}{q_m}\dot{q}^m
\end{equation}
luego
\begin{equation}
  [L]_n=\pdv{L}{q_n}-\pdv{L}{\dot{q}_n}{q_m}\dot{q}^m-\pdv{L}{\dot{q}_n}{\dot{q}_m}\ddot{q}^m=0
\end{equation}
así
\begin{equation}
  \boxed{\left(\pdv{L}{\dot{q}_n}{\dot{q}_m}\right)\ddot{q}^m=\pdv{L}{q_n}-\pdv{L}{\dot{q}_n}{q_m}\dot{q}^m}
\end{equation}
Notemos que para poder expresar $\ddot{q}$ como función de $q$ y $\dot{q}$ es necesario que la matriz $W_{nm}=\pdv*{L}{\dot{q}_n}{\dot{q}_m}$ sea invertible, es decir, $\det W_{nm}\neq 0$.

Llamando 
\begin{equation}
V_n=\pdv{L}{q_n}-\pdv{L}{\dot{q}_n}{q_m}{\dot{q}^m}
\end{equation}
tenemos
\begin{equation}\label{2.2}
  W_{nm}\ddot{q}^{m}-V_n=0
\end{equation}
Si $\det W_{nm}\neq 0$ entonces existe una matriz inversa $W^{kn}\equiv (W_{kn})^{-1}$ tal que $W^{kn}W_{nm}=\delta ^k_m$. Luego, multiplicando \eqref{2.2} por $W^{km}$, tenemos
\begin{align}
 & W^{kn}W_{nm}\ddot{q}^m-W^{kn}V_n=0\\
  \Rightarrow \ & \ddot{q}^k=W^{kn}V_n=F(q,\dot{q},t)
\end{align}
En la física fundamental, las teorías de gauge tales como la teoría electromagnética o las toerías de Yang-Mills (teoría electrodébil, cromonodinámica cuántica), las correspondientes funciones de Lagrange tienen sus matrices Hessianas singulares, es decir, $\det W_{nm}\neq 0$.

\subsection{Formalismo de Hamilton}
Consiste en pasarse de las coordenadas $\{q_i,\dot{q}_i,t\}$ a $\{q_i,p_i,t\}$, donde
\begin{equation}\label{2.3}
  p_i=\pdv{L}{\dot{q}_i}=f_i(q_i,\dot{q}_i,t)
\end{equation}
es el momentum generalizado.

Para escribir explícitamente la función de Hamilton es necesario expresar por medio de \eqref{2.3} $\dot{q}_i=\bar{f}(q_i,p_i)$. Esto implica que la función $f_i$ sea invertible,
\begin{equation}
  \dot{q}_n\to p_n=f_n(q_m,\dot{q}_m,t)
\end{equation}
es decir, tenemos una transformación de coordenadas. Esta transformación tiene como matriz Jacobiana a 
\begin{equation}
  J_{nm}=\pdv{f_n}{\dot{q}_m}=\pdv{p_n}{\dot{q}_m}=\pdv{\dot{q}_m}\left(\pdv{L}{\dot{q}_n}\right)
\end{equation}
esto es
\begin{equation}
  \boxed{J_{nm}=\pdv{L}{\dot{q}_n}{\dot{q}_m}\equiv W_{nm}}
\end{equation}

Para clarificar esto calculemos $\dd p_n$ recordando que $p_n=f_n(q_m,\dot{q}_m,t)$,
\begin{align}
  \dd p_n&=\pdv{f_n}{t}\dd t+\pdv{f_n}{q_m}\dd q_m+\pdv{f_n}{\dot{q}_m}\dd \dot{q}_m\\
  &=\pdv{f_n}{t}\dd t+\pdv{f_n}{q_m}\dd q_m+\pdv{L}{\dot{q}_n}{q_m}\dd \dot{q}_m
\end{align}
esto implica que
\begin{equation}
  \left(\pdv{L}{\dot{q}_n}{q_m}\right)\dd \dot{q}_m=\dd p_n-\pdv{p_n}{t}\dd t-\pdv{p_n}{q_m}\dd q_m
\end{equation}
De aquí vemos que para expresar $\dot{q}=\bar{f}(q,p,t)$ es necesario que
\begin{equation}
  \det W_{nm}=\det \left(\pdv{L}{\dot{q}_n}{q_m}\right)\neq 0
\end{equation}


\subsection{*Transformaciones canónicas}
Son transformaciones invertibles de la forma (Ref. \cite{Schwichtenberg:2018dri}) \footnote{Ver también el libro Arnold. y \cite{book:1007830} }
\begin{equation}
  \hat{q}^j=\hat{q}^j(q,p),\qquad \hat{p}^j=\hat{p}^j(q,p)
\end{equation}
que dejan los corchetes fundamentales invariantes.

Antes de continuar, introduzcamos una notación más compacta en la cual colectamos las $2N$ variables del espacio de fase en un único conjunto $(x^\alpha)=(q^1,...,q^N,p_1,...,p_N)$. En esta notación los corchetes fundamentales pueden ser escritos como
\begin{equation}
  \{x^\alpha,x^\beta\}=\Gamma^{\alpha \beta},\qquad \mbox{con }\quad \Gamma\equiv \mqty(0_N&1_N\\-1_N&0_N)
\end{equation}
en términos de la matriz $\Gamma$, el corchete de Poisson para dos funciones del espacio de fase $A$ y $B$ queda
\begin{equation}
  \{A,B\}=\Gamma^{\alpha \beta}\pdv{A}{x\alpha}\pdv{B}{x^\beta}
\end{equation}
La condición para que $\hat{x}(x)$ sea una transformación canónica se simplifica a $\{\hat{x}^\alpha,\hat{x}^\beta\}=\Gamma^{\alpha \beta}$























































