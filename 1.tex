\section{Clase 1}
\subsection{Mecánica de Newton}
Posición, velocidad, aceleración, fuerza.

Si consideramos un sistema de partículas de masa $m$
\begin{equation}
  \vec{p}_\alpha =m_\alpha \vec{\dot{x}}_\alpha,\qquad \vec{p}=\sum_\alpha \vec{p}_\alpha,\qquad \alpha=1,...,k
\end{equation}
\begin{equation}
  T=\frac{1}{2}\sum_\alpha m_\alpha \vec{\dot{x}}^2,\qquad \vec{L}=\sum_\alpha \vec{x}_\alpha\times \vec{p}_\alpha
\end{equation}
Newton estableció que a dinámica de un sistema mecánico queda	determinada por tres leyes fundamentales.






\subsection{Acerca de la matriz Hessiana de las ecuaciones de Euler-Lagrange}
\begin{equation}
  \dv{t}\pdv{L'}{\dot{q}'_k}=\pdv{L'}{q'_k}-[L]_k\pdv{q_l}{\dot{q}_k}
\end{equation}
\begin{equation}
  \Rightarrow \pdv{L'}{q'_k}=\dv{t}\pdv{L'}{\dot{q}_k}=[L]_l\pdv	{q_l}{q'_k}
\end{equation}
\begin{equation}
\boxed{[L']_k=[L]_l\pdv{q_l}{q'_k}}
\end{equation}
La derivada de Euler-Lagrange transforma como un vector covariante bajo una transformación de coordenadas
\begin{equation}
  \mbox{Si }[L]_l=0 \Rightarrow [L']_k=0
\end{equation}


