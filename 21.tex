\section{Clase 21}
Hasta ahora tenemos:
\begin{enumerate}
	\item Electrodinámica, con su simetría de gauge. Esto implica que las ecuaciones de Maxwell son invariantes bajo
	\begin{equation}\label{21.1}
	\begin{split}
  \vec{A}&\to \vec{A}'=\vec{A}+\nabla\chi \\
  V&\to V'=V-\pdv{\chi}{t}
  \end{split} 
\end{equation}
\item La mecánica cuántica es invariante bajo la transformación
 \begin{equation}\label{21.2}
  \psi\to \psi'=\psi e^{i\a }
\end{equation}
El conjunto de las transformaciones \eqref{21.2} forman el grupo abeliano unitario $U(1)$. ($U(1)$ global).

\item De acuerdo a Weyl, Schrodinger y London, las simetría debería ser locales. Es decir, deberiamos generalizar  la transformaciones \eqref{21.2} al caso local
\begin{equation}\label{21.3}
  \psi\to \psi'=\psi e^{i\a(x) }
\end{equation}
El conjunto de las transformaciones \eqref{21.3} constituyen el grupo $U(1)$ local.


\end{enumerate}
\underline{Pregunta:} ¿Existe alguna relación entre las transformaciones \eqref{21.1} y \eqref{21.3} ?

Consideremos la ecuación de Schrodinger para una partícula en un campo electromagnético. La partícula material representada por la onda de materia de De-Broglie es descrita por la ecuación de Schrodinger. El campo electromagnético es decrito por las ecuaciones de Maxwell. ¿Son compatibles las ecuaciones de Maxwell con la mecánica cuántica?

La ecuación de Schrodinger para la partícula libre es 
\begin{equation}
  -\frac{\hbar^2}{2m}\nabla^2\psi =i\hbar\pdv{\psi}{t}
\end{equation}
Para $\hbar=1$,
\begin{equation}
  -\frac{1}{2m}\nabla^2\psi=i\pdv{\psi}{t}
\end{equation}
\begin{equation}\label{21.Sc}
  \frac{1}{2m}(-i\nabla)^2\psi=i\pdv{\psi}{t}
\end{equation}
Para obtener la ecuación de Schrodinger para una partícula en un campo electromagnético basta con sustituir:
\begin{equation}
  -i\nabla\to -i\nabla -q\vec{A}=-i\left(\nabla-iq\vec{A}\right)
\end{equation}
\begin{equation}
  i\pdv{t}\to i\pdv{t}-qV=i\left(\pdv{t}+iqV\right)
\end{equation}
Luego, \eqref{21.Sc} queda
\begin{equation}\label{21.Sc2}
  \frac{1}{2m}\left[-i(\nabla-iq\vec{A})\right]^2\psi\xt =
  i\left(\pdv{t}+iqV\right)\psi\xt 
\end{equation}
Definiendo:
\begin{align}
  \vec{D}&=\nabla -iq\vec{A}\\
  D^0&=\pdv{t}+iqV
\end{align}

\eqref{21.Sc2} queda
\begin{equation}\label{21.4}
  \boxed{\frac{1}{2m}(-i\vec{D})^2\psi\xt =iD^0\psi\xt }
\end{equation}
Esta ecuación describe a una partícula en un campo electromagnético.

Estudiemos la invariancia de \eqref{21.4} bajo de gauge \eqref{21.1}, 
\begin{equation}
  \frac{1}{2m}\left[-i(\nabla-iq\vec{A}')\right]^2\psi\xt =i\left(\pdv{t}+iqV'\right)\psi\xt 
\end{equation}
\begin{equation}\label{21.5}
  \frac{1}{2m}\left[-i(\nabla-iq\vec{A}-iq\nabla\chi )\right]^2\psi\xt =i\left(\pdv{t}+iqV-iq\pdv{\chi }{t}\right)\psi\xt 
\end{equation}

La ecuación \eqref{21.5} muestra una aparente incompatibilidad entre la electrodinámica y la mecánica cuántica.

\begin{teor}
	La ecuación de Schrodinger que describe una partícula de carga $q$ en un campo electromagnético
	\begin{equation}
  \frac{1}{2m}\left[-i(\nabla-iq\vec{A})\right]^2\psi\xt =
  i\left(\pdv{t}+iqV\right)\psi\xt 
\end{equation}
es invariante bajo las transformaciones
	\begin{equation}
	\begin{split}
  \vec{A}&\to \vec{A}'=\vec{A}+\nabla\chi \\
  V&\to V'=V-\pdv{\chi}{t}\\
  \psi\xt &\to \psi'\xt =e^{iq\chi\xt }\psi\xt 
  \end{split} 
\end{equation}
\end{teor}

\begin{prueba}
	Sabemos que una partícula en un campo electromagnético es regida por la ecuación 
	\begin{equation}
  \frac{1}{2m}(-i\vec{D})^2\psi\xt =iD^0\psi\xt 
\end{equation}
Esta ecuación se puede reescribir como
\begin{equation}\label{21.6}
  \frac{1}{2m}(-i\vec{D}')^2\psi'\xt =iD'^0\psi'\xt 
\end{equation}
Para el lado izquerdo, se tiene
\begin{align}
  -i\vec{D}'\psi'&=-i(\nabla-iq\vec{A}')\psi'\\
  &=(-i\nabla-q\vec{A}-q\nabla\chi )(e^{iq\chi }\psi)\\
  &=-i\nabla(e^{iq\chi }\psi)-q\vec{A}e^{iq\chi }\psi-q\nabla\chi e^{iq\chi }\psi\\
  &=-i^2q\nabla\chi e^{iq\chi }\psi-ie^{iq\chi }\nabla\psi -q\vec{A}e^{iq\chi }\psi	-q\nabla\chi e^{iq\chi }\psi\\
  &=-ie^{iq\chi }\nabla\psi-q\vec{A}e^{iq\chi }\psi\\
  &=e^{iq\chi }(-i\nabla-q\vec{A})\psi\\
  &=e^{iq\chi }(-i(\nabla-iq\vec{A}))\psi
\end{align}
Es decir,
\begin{equation}\label{21.7}
  \boxed{-i\vec{D}'\psi'=e^{iq\chi }(-i\vec{D})\psi}
\end{equation}

Para el lado derecho de \eqref{21.6} se tiene
\begin{align}
  iD'^0\psi'\xt &=i\left(\pdv{t}+iqV'\right)\psi'\\
  &=i\left(\pdv{t}+iqV-iq\pdv{\chi}{t}\right)(e^{iq\chi }\psi)\\
  &=i\pdv{t}(e^{iq\chi }\psi )-qVe^{iq\chi }\psi +q\pdv{\chi}{t}e^{iq\chi }\psi\\
  &=i^2q\pdv{\chi}{t}e^{iq\chi }\psi +ie^{iq\chi }\pdv{\psi}{t}-qVe^{iq\chi }\psi+q\pdv{\chi }{t}e^{iq\chi }\psi\\
  &=ie^{iq\chi }\pdv{\psi}{t}-qVe^{iq\chi }\psi \\
  &=e^{iq\chi }\left(i\pdv{t}-qV\right)\psi \\
  &=e^{iq\chi }\left[i\left(\pdv{t}+iqV\right)\right]\psi
\end{align}
De manera que
\begin{equation}\label{21.8}
\boxed{   iD'^0\psi' =e^{iq\chi }(iD^0\psi )}
\end{equation}

De \eqref{21.4} y \eqref{21.8}, 
\begin{align}\label{21.9}
  iD^0\psi'=e^{iq\chi }\left(\frac{1}{2m}(-i\vec{D})^2\psi \right)
\end{align}
De \eqref{21.6} y \eqref{21.9},
\begin{align}\label{21.10}
  \frac{1}{2m}(-i\vec{D}')^2\psi=e^{iq\chi }\left(\frac{1}{2m}(-i\vec{D})^2\psi\right)
\end{align}
Luego,
\begin{equation}
  e^{iq\chi }\left(\frac{1}{2m}(-i\vec{D})^2\psi \right)=e^{iq\chi }(iD^0\psi )
\end{equation}
\begin{equation}
  \implies \frac{1}{2m}(-i\vec{D})^2\psi =iD^0 \psi \qquad \qed 
\end{equation}
\end{prueba}

\subsection{Electrodinámica a partir del principio de gauge}
Antes de estudiar esto estudiaremos:
\begin{itemize}
	\item La ecuación de Schrodinger
	\item La ecuación de Klein-Gordon
	\item La ecuación de Dirac
\end{itemize}

\subsection{La ecuación de Schrodinger}
El Hamiltoniano para una partícula libre es
\begin{equation}
H=  \frac{p^2}{2m}
\end{equation}
donde $H=E$.

El paso a la mecánica cuántica es $H\to \hat{H}$, $E\to \hat{E}$. Así, se tiene
\begin{equation}
  \hat{H}=\frac{\hat{p}}{2m}
\end{equation}
donde $\hat{p}=-i\hbar\nabla$ y $\hat{E}=i\hbar\pdv{t}$. Luego,
\begin{align}
  \hat{H}&=\frac{1}{2m}\hat{p}\cdot\hat{p}\\
  &=\frac{1}{2m}(-i\hbar\nabla)(-i\hbar\nabla)\\
  &=-\frac{\hbar^2}{2m}\nabla^2
\end{align}
\begin{equation}
  \implies \hat{H}\psi=\hat{E}\psi\implies \boxed{-\frac{\hbar^2}{2m}\nabla^2\psi=i\hbar\pdv{\psi}{t}}
\end{equation}
Esta ecuación no es invariante de Lorentz, luego, no es válida para partículas con alta energía, es decir, con alta velocidad\footnote{En átomos livianos $v\sim 10^3$ km/s mientras que en átomos pesados $v\sim 10^5$ km/s.}.

\subsection{Ecuación de Klein-Gordon}
También conocida como \textit{ecuación de Schrodinger relativista}. 

De la teoría Especial de la Relatividad sabemos que
\begin{equation}
  p^\m =\left(\frac{E}{c},\vec{p}\right),\qquad p_\m =\left(\frac{E}{c},-\vec{p}\right)
\end{equation}
y además
\begin{equation}
  p_\m p^\m =m_0^2c^2
\end{equation}
Así,
\begin{align}
  p_\m p^\m &=\left(\frac{E}{c},-\vec{p}\right)\left(\frac{E}{c},\vec{p}\right)\\
  &=\frac{E^2}{c^2}-p^2
\end{align}
\begin{equation}
  \implies \frac{E^2}{c^2}-p^2=m_0^2c^2
\end{equation}
\begin{equation}\label{21.E2}
  \implies \boxed{E^2=p^2c^2+m_0^2c^4}
\end{equation}
El paso a la mecánica cuántica:
\begin{equation}
  \hat{p}_\m =i\hbar \partial_\m ,\qquad \hat{p}^\m =i\hbar\partial^\m 
\end{equation}
\begin{align}
  \hat{p}_\m =\left(\frac{\hat{E}}{c}-\hat{\vec{p}}\right)=\left(\frac{i\hbar}{c}\pdv{t},i\hbar\nabla\right)=i\hbar\left(\pdv{x^0},\nabla \right)\
\end{align}
\begin{equation}
  \hat{p}_\m =i\hbar\partial_\m =i\hbar\left(\pdv{x^0},\nabla \right)
\end{equation}
\begin{equation}
  \hat{p}^\m =i\hbar\partial^\m =i\hbar\left(\pdv{x^0},-\nabla \right)
\end{equation}
\begin{equation}
  \implies \hat{p}_\m \hat{p}^\m \psi =m_0^2c^2\psi=(i\hbar\partial_\m )(i\hbar\partial^\m )\psi 
\end{equation}
de manera que la ecuación de Klein-Gordon queda
\begin{equation}\label{21.KG}
  \boxed{\partial_\m \partial^\m \psi+\frac{m_0^2c^2}{\hbar^2}\psi=0}
\end{equation}

Notemos que podemos llegar a esta ecuación de otra manera. Consideremos \eqref{21.E2}
\begin{equation}
  E^2=p^2c^2+m_0^2c^4
\end{equation}
Promoviendo a operadores,
\begin{equation}
  \hat{E}^2=\hat{p}^2c^2+m_0^2c^4
\end{equation}
\begin{equation}
  \left(i\hbar\pdv{t}\right)\left(i\hbar\pdv{t}\right)\psi=c^2(-i\hbar\nabla)(-i\hbar\nabla)\psi +m_0^2c^4\psi 
\end{equation}
\begin{equation}
  -\hbar^2\pdv[2]{\psi}{t}=-\hbar^2c^2\nabla^2\psi +m_0^2c^4\psi
\end{equation}
\begin{equation}
  \frac{1}{c^2}\pdv[2]{\psi}{t}=\nabla^2\psi-\frac{m_0^2c^4}{\hbar^2c^2}
\end{equation}
\begin{equation}
  \implies\boxed{\partial_\m \partial^\m \psi+\frac{m_0^2c^2}{\hbar^2}\psi=0}
\end{equation}
Obteniendo nuevamente \eqref{21.KG}.

Consideremos ahora la densidad y corriente de probabilidad de la ecuación de Klein-Gordon y su compleja comjugada:
\begin{align}
  \partial_\m \partial^\m \psi+\frac{m_0^2c^2}{\hbar^2}\psi &=0\\
  \partial_\m \partial^\m \psi^*+\frac{m_0^2c^2}{\hbar^2}\psi^*&=0
\end{align}

\begin{align}
  \frac{\hbar}{m_0}\partial_\m \partial^\m \psi+\frac{m_0c^2}{\hbar}\psi & =0 \qquad /\frac{i}{2}\psi^*\cdot ()\\
  \frac{\hbar}{m_0}\partial_\m \partial^\m \psi^*+\frac{m_0c^2}{\hbar}\psi^* & =0\qquad /\frac{i}{2}\psi\cdot ()
\end{align}
\begin{align}
  \frac{i\hbar}{2m_0}\psi^*\partial_\m \partial^\m \psi +\frac{im_0c^2}{2\hbar}\psi\psi^*&=0\label{21.11}\\
  \frac{i\hbar}{2m_0}\psi\partial_\m \partial^\m \psi^* +\frac{im_0c^2}{2\hbar}\psi^*\psi&=0\label{21.22}
\end{align}
Restando \eqref{21.22} de \eqref{21.11},
\begin{align}
  \frac{i\hbar}{2m_0}\left(\psi^*\partial_\m \partial^\m \psi-\psi\partial_\m \partial^\m \psi^*\right)&=0\\
  \frac{i\hbar}{2m_0}\left[\partial_\m (\psi^*\partial^\m \psi)-\partial_\m\psi^*\partial^\m \psi-\partial_\m (\psi\partial^\m \psi^*)+\partial_\m \psi\partial^\m \psi^*\right]&=0\\
  \partial_\m \left[\frac{i\hbar}{2m_0}(\psi^*\partial^\m \psi-\psi\partial^\m \psi^*)\right]&=0\\
  \partial_\m J^\m &=0
\end{align}
con
\begin{equation}
  J^\m =\frac{i\hbar}{2m_0}(\psi^*\partial^\m \psi-\psi\partial^\m \psi^*)
\end{equation}

Recordando que
\begin{equation}
  \partial_\m =\left(\frac{1}{c}\pdv{t},\nabla\right),\qquad \partial^\m =\left(\frac{1}{c}\pdv{t},-\nabla\right)
\end{equation}
tenemos
\begin{align}
  \left(\frac{1}{c}\pdv{t},\nabla\right)\left[\frac{i\hbar}{2m_0}\left(\psi^*\left(\frac{1}{c}\pdv{t},-\nabla\right) \psi-\psi\left(\frac{1}{c}\pdv{t},-\nabla\right) \psi^*\right)\right]&=0\\
  \left(\frac{1}{c}\pdv{t},\nabla\right)\left[\frac{i\hbar}{2m_0c}\left(\psi^*\pdv{\psi}{t}-\psi\pdv{\psi^*}{t}\right),-\frac{i\hbar}{2m_0}\left(\psi^*\nabla\psi-\psi\nabla\psi^*\right)\right]&=0\\
  \pdv{t}\left[\frac{i\hbar}{2m_0c^2}\left(\psi^*\pdv{\psi}{t}-\psi\pdv{\psi^*}{t}\right)\right]+\nabla\left[-\frac{i\hbar}{2m_0}\left(\psi^*\nabla\psi-\psi\nabla\psi^*\right)\right]&=0
\end{align}
lo que implica que
\begin{equation}
  \pdv{\rho }{t}+\nabla\cdot\vec{J}=0
\end{equation}
donde
\begin{align}
  \rho&=\frac{i\hbar}{2m_0c^2}\left(\psi^*\pdv{\psi}{t}-\psi\pdv{\psi^*}{t}\right)\\
  \vec{J}&=-\frac{i\hbar}{2m_0}\left(\psi^*\nabla\psi-\psi\nabla\psi^*\right)
\end{align}







































































