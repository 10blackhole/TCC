\section{Clase 14}
Habíamos visto que

\begin{equation}
  (\Lambda,a)\to g(\Lambda,a)
\end{equation}
\begin{equation}
  g(\Lambda_2,a_2)g(\Lambda_1,a_1)=g(\Lambda_2\Lambda_1,\Lambda_2a_1+a_2)
\end{equation}
\begin{equation}
  g^{-1}(\Lambda,a)=g(\Lambda^{-1},-\Lambda^{-1}a)
\end{equation}
\subsection{Algebra de Poincare}
Consideremos el siguiente producto

\begin{equation}\label{14.prod}
  g^{-1}(\Lambda,0)g(\Lambda',a')g(\Lambda,0)=?
\end{equation}

Notemos que
\begin{equation}
  g^{-1}(\Lambda,0)=g(\Lambda^{-1},0)
\end{equation}
además
\begin{align}
  g(\Lambda',a')g(\Lambda,0)&=g(\Lambda'\Lambda,\Lambda'\cdot 0+a')\\
  &=g(\Lambda'\Lambda,a')
\end{align}
luego \eqref{14.prod} queda
\begin{align}
  g^{-1}(\Lambda,0)g(\Lambda',a')g(\Lambda,0)&=g(\Lambda^{-1},0)g(\Lambda'\Lambda,a')\\
  &=g(\Lambda^{-1}\Lambda'\Lambda,\Lambda^{-1}a')
\end{align}
\begin{equation}\label{14.1}
  \boxed{ g^{-1}(\Lambda,0)g(\Lambda',a')g(\Lambda,0)=g(\Lambda^{-1}\Lambda'\Lambda,\Lambda^{-1}a')}
\end{equation}

Para calcular el álgebra consideramos el caso infinitesimal,
\begin{equation}
  g(\Lambda',a')=\id -\frac{i}{2}\omega'_{\rho \sigma }M^{\rho\s }+ia'_\m P^\m 
\end{equation}

Estudiemos el lado izquierdo de \eqref{14.1},
\begin{align}
  g^{-1}(\Lambda,0)g(\Lambda',a')g(\Lambda,0)&=g^{-1}(\Lambda,0)\left[\id -\frac{i}{2}\omega'_{\m  \n  }M^{\m \n  }+ia'_\rho P^\rho \right]g(\Lambda,0)\\
  &=\id-\frac{i}{2}\omega'_{\m\n }g^{-1}(\Lambda,0)M^{\m\n }g(\Lambda,0)+ia'_\rho g^{-1}(\Lambda,0)P^\rho g(\Lambda,0)\label{14.2}
\end{align}
Para el lado derecho de \eqref{14.1}
\begin{align}
  g(\Lambda^{-1}\Lambda'\Lambda,\Lambda^{-1}a')&=\id 
\end{align}

%TODO terminar el cálculo
























