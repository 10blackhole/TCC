\section{Clase 5}\label{clase:5}
Distinguimos entre dos tipos de variaciones. La primera es la \textbf{variación $\d$} la cual compara dos coordenadas distintas en tiempos distintos. Es decir, compara las coordenadas $q$ y $\qp$ en los tiempos $t$ y $t'$,
\begin{equation}
  \delta q=\qp(t')-q(t),\qquad t'=t+\d t
\end{equation}
\begin{equation}
  \qp(t')=q(t)+\d t
\end{equation}
Por otro lado la \textbf{variación $\bar{\delta}$} compara dos coordenadas distintas en el mismo instante. Es decir, compara las coordenadas $q$ y $\qp$ en el mismo instante,
\begin{equation}
  \db q=\qp(t)-q(t)
\end{equation}
\begin{equation}
  \qp(t)=q(t)+\db q
\end{equation}
\subsection{Relación entre  y }
\begin{align}
  \db q&=\qp(t)-q(t)+\qp(t')-\qp(t')\\
  &=(\qp(t')-q(t))+\qp(t)-\qp(t')
\end{align}
\begin{equation}\label{5.1.1}
  \implies \db q=\d q-[\qp(t')-\qp(t)]
\end{equation}
Pero
\begin{equation}
  \qp(t')=\qp(t+\d t)=\qp(t)+\d t\dv{\qp(t)}{t}
\end{equation}
\begin{align}
  \implies \qp(t')-\qp(t)&=\d t\dv{t}(q(t)+\db q)\\
  &=\d t\dv{t}q(t)+\cancel{\d t\dv{t}\db q}
\end{align}
el ultimo término es despreciable por que es de segundo orden.
\begin{equation}
  \implies \qp(t')-\qp(t)=\d t\qd
\end{equation}
\begin{equation}
  \implies\boxed{ \db q=\d q(t)-\d t\qd(t)}
\end{equation}

\begin{prop}
	Los operadores $\db$ y $\dv*{t}$ conmutan.
\end{prop}
\begin{align}
  \db \qd&=\qdp(t)-\qd(t)=\dv{t}\qp(t)-\qd(t)\\
  &=\dv{t}\left(q(t)-\db q\right)-\qd(r)\\
  &=\cancel{\qd(t)}+\dv{t}\db q-\cancel{\qd(t)}
\end{align}
\begin{equation}
  \implies \boxed{\db\qd=\dv{t}\db q},\qquad\implies  \boxed{\left[\db,\dv{t}\right]=0}
\end{equation}

\begin{prop}
	Los operadores $\d$ y $\dv*{t}$ no conmutan.
\end{prop}

\subsection{Prueba del teorema de Noether
}
Hemos visto que
\begin{itemize}
	\item \begin{equation}
  S'(\qp,\qdp,t')=S(q,\qd,t)
\end{equation}

	\item \begin{equation}\label{5.1}
  L'(q',\qdp,t')=L[q_i(q',t),\qd_i(q,\qdp,t'),t(t')]\dttp
\end{equation}

	\item \begin{equation}\label{5.2}
  L'(\qp,\qdp,t')=L(\qp,\qdp,t')+\dv{t'}\Omega(q',t')
\end{equation}
\end{itemize}
De \eqref{5.1} y \eqref{5.2} vemos
\begin{equation}
  L[q(\qp,t'),\qd(\qp\qdp,t'),t(t')]\dttp=L(\qp,\qdp,t')\dv{t'}\Omega(\qp,t')
\end{equation}
cambiando a las coordenadas antiguas,
\begin{equation}
  L(q,\qd,t)=L[\qp(q,t),\qdp(q\qd,t),t'(t)]\dv{t'}{t}+\dv{t'}\Omega(q'(q,t),t'(t))\dv{t'}{t}
\end{equation}
\begin{equation}
  L(q,\qd,t)=L(\qp\qdp,t')\dv{t'}{t}+\dv{t}\Omega(\qp,t')
\end{equation}
en el entendido que
\begin{equation}
  \qp=\qp(q,t),\qquad \qdp=\qdp(q,\qd,t),\qquad t'=t'(t)
\end{equation}
Dado que $t'=t+\d t$,
\begin{equation}
  \dtpt=1+\dv{t}\d t
\end{equation}
\begin{equation}
  L(q,\qd,t)=L(\qp,\qdp,t')\left(1+\dv{t}\d t\right)+\dv{t}\Omega(\qp,t')
\end{equation}
\begin{equation}\label{5.flecha}
  \implies L(q,\qd,t)-L(\qp,\qdp,t')=L(\qp,\qdp,t')\dv{t}\d t+\dv{t}\Omega(\qp,t')
\end{equation}
Dado que las transformaciones son continuas, basta estudiar el caso infinitesimal.

 De \eqref{5.flecha},
\begin{align}
  -\d L&=L(q,\qd,t)-L(q+\d q,\qd+\d\qd,t+\d t)\\
  &=L(q+\d q,\qd+\d\qd,t+\d t)\dv{t}\d t+\dv{t}\Omega(q+\d q,t+\d t)
\end{align}
Expandiendo el primer término hasta primer orden
\begin{equation}
  -\d L=L(q,\qd,t)\dv{t}\d t+\dv{t}\Omega(q+\d q,t+\d t)
\end{equation}
Si consideramos el caso límite donde $\d q=0, \d t=0$
\begin{equation}
  \d L=0,\qquad \dv{t}\Omega(q,t)=0
\end{equation}
Esto nos permite escribir\begin{align}
  -\d L&=L(q,\qd,t)\dv{t}\d t+\dv{t}\Omega(q+\d q,t+\d t)-\dv{t}\Omega(q,t)\\
  &=L(q,\qd,t)\dv{t}+\d t\dv{t}[\Omega(q+\d q,t+\d t)-\Omega(q,t)]
\end{align}
\begin{equation}\label{5.3}
  \implies \boxed{-\d L=L(q,\qd,t)\dv{t}\d t+\dv{t}\d \Omega(q,t)}
\end{equation}
Reemplazando $\Lag$, tenemos
\begin{equation}\label{5.4}
  \d L=\sum_i\left(\pdv{L}{q_i}\d q_i+\pdv{L}{\qd_i}\d \qd_i\right)+\pdv{L}{t}\d t
\end{equation}
Reemplazando \eqref{5.4} en \eqref{5.3},
\begin{equation}\label{5.5}
  -\sum_i\left(\pdv{L}{q_i}\d q_i+\pdv{L}{\qd_i}\d \qd_i\right)-\pdv{L}{t}\d t=L(q,\qd,t)\dv{t}\d t+\dv{t}\d \Omega(q,t)
\end{equation}

Estudiaremos ahora el primer término del lado izquierdo. Dado que $\d\qd_i=\dv*{t}\d q_i-\qd_i\dv*{t}\d t$ (eucación \eqref{clase5-4}),
\begin{align}
  \pdv{L}{\qd_i}\d \qd_i&=\pdv{L}{\qd_i}\left(\dv{t}\d q_i-\qd_i\dv{t}\d t\right)\\
  &=\pdv{L}{\qd_i}\dv{t}\d q_i-\pdv{L}{\qd_i}\qd_i\dv{t}\d t
\end{align}
\begin{align}
  \implies \sum_i\pdv{L}{q_i}\d q_i+\pdv{L}{\qd_i}\d \qd_i&=\sum_i\pdv{L}{q_i}\d q_i+\pdv{L}{\qd_i}\dv{t}\d q_i-\pdv{L}{\qd_i}\qd_i\dv{t}\d t\\
  &=\sum_i\left(\pdv{L}{q_i}+\pdv{L}{\qd_i}\dv{t}\right)\d q_i-\sum_i\pdv{L}{\qd_i}\qd_i\dv{t}\d t
\end{align}
Introduciendo en \eqref{5.5},
\begin{equation}
  \sum_i\left(\pdv{L}{q_i}+\pdv{L}{\qd_i}\dv{t}\right)\d q_i-\sum_i\pdv{L}{\qd_i}\qd_i\dv{t}\d t+\pdv{L}{t}\d t=-L(q,\qd,t)\dv{t}\d t-\dv{t}\d \Omega(q,t)
 \end{equation}
 \begin{equation}
  \sum_i\left(\pdv{L}{q_i}+\pdv{L}{\qd_i}\dv{t}\right)\d q_i+\left(L(q,\qd,t)-\sum_i\pdv{L}{\qd_i}\qd_i\right)\dv{t}\d t+\pdv{L}{t}\d t=-\dv{t}\d \Omega(q,t)
\end{equation}
después de algo de cálculo se llega a 
%TODO se podria completar
\begin{equation}\label{5.6}
\begin{split}
  \dv{t}\left[\sum_i\pdv{L}{\qd_i}\d q_i+L\d t-\sum_i\pdv{L}{q_i}\qd_i\d t+\d \Omega\right]=-\sum_i\left(\pdv{L}{q_i}-\dv{t}\pdv{L}{\qd_i}\right)\d q_i \\ +\left(\dv{L}{t}-\pdv{L}{t}\right)\d t-\sum_i\dv{t}\left(\pdv{L}{\qd_i}\qd_i\right)\d t
 \end{split}
\end{equation}
Analicemos los dos últimos términos de \eqref{5.6}. Dado que $\Lag$, se tiene que
\begin{equation}
  \dv{L}{t}=\sum_i\left(\pdv{L}{q_i}\qd_i+\pdv{L}{q_i}\ddot{q}_i\right)+\pdv{L}{t}
\end{equation}
\begin{equation}
  \implies \dv{L}{t}-\pdv{L}{t}=\sum_i\left(\pdv{L}{q_i}\qd_i+\pdv{L}{q_i}\ddot{q}_i\right)
\end{equation}
y además
\begin{equation}
  \sum_i\dv{t}\left(\pdv{L}{q_i}\qd_i\right)=\sum_i\left(\dv{t}\pdv{L}{\qd_i}\right)\qd_i+\pdv{L}{q_i}\ddot{q}_i
\end{equation}
\begin{align}
  \left(\dv{L}{t}-\pdv{L}{t}\right)-\sum_i\dv{t}\l(\pdv{L}{\qd_i}\qd_i\r)&=\sum_i\l(\pdv{L}{q_i}\qd_i+\cancel{\pdv{L}{q_i}\ddot{q}_i}-\l(\dv{t}\pdv{L}{\qd_i}\r)\qd_i-\cancel{\pdv{L}{q_i}\ddot{q_i}}\r)\\
  &=\sum_i\l(\pdv{L}{q_i}-\dv{t}\pdv{L}{\qd_i}\r)\qd_i\label{5.7}
\end{align}
Así, \eqref{5.6} toma la forma
\begin{equation}
  \dv{t}\left[\sum_i\pdv{L}{\qd_i}\l(\d q_i-\qd_i\d t\r)+L\d t+\d\Omega\right]=-\sum_i\l(\pdv{L}{q_i}-\dv{t}\pdv{L}{\qd_i}\r)(\d q_i-\qd_i\d t)
\end{equation}
Dado que
\begin{equation}
  \db q_i=\d q_i-\qd_i\d t
\end{equation}
y que
\begin{equation}
  [L]_i=\sum_i\l(\EL\r)
\end{equation}
tenemos
\begin{equation}
  \dv{t}\left[\pdv{L}{\qd_i}\l(\d q_i-\qd_i\d t\r)+L\d t+\d\Omega\right]=-[L]_i\db q_i
\end{equation}

luego,
\begin{equation}
  [L]_i\db q_i+\dv{t}\left[\sum_i\pdv{L}{\qd_i}\db q_i+L\d t+\d\Omega \right]=0
\end{equation}
esto implica, que en el espacio de soluciones de las ecuaciones de Euler-Lagrange, se tiene
\begin{equation}
  \dv{t}\left[\sum_i\pdv{L}{\qd_i}\db q_i+L\d t+\d\Omega \right]=0
\end{equation}
Definiendo la cantidad,
\begin{equation}
  J\equiv \sum_i\pdv{L}{\qd_i}\db q_i+L\d t+\d\Omega 
\end{equation}
\begin{equation}
  \implies \dv{J}{t}=0,\implies   J=\text{constante}
\end{equation}
Luego, $J$ es una cantidad conservada,
\begin{equation}
  \boxed{[L]_i\db q_i+\dv{J}{t}=0}
\end{equation}






























