\section{Clase 23}
\subsection{Cuadri-corriente y ley de conservación}
Consideremos la ecuación de Dirac y su conjugada
\begin{equation}
  i\hbar\pdv{\psi}{t}=\frac{\hbar c}{i}\sum_k\a_k\pdv{\psi}{x^k }+\b m_0c^2\psi\qquad /\psi^\dagger\cdot
\end{equation}
\begin{equation}
  -i\hbar\pdv{\psi^\dagger}{t}=\frac{\hbar c}{-i}\sum_k\a_k\pdv{\psi^\dagger}{x^k }+m_0c^2\psi^\dagger\b^\dagger\qquad /\cdot\psi
\end{equation}
\begin{equation}\label{23.a}
  \implies i\hbar\psi^\dagger \pdv{\psi}{t}=\frac{\hbar c}{i}\sum_k\psi^\dagger \a_k\pdv{\psi}{x^k }+m_0c^2\psi^\dagger \b\psi
\end{equation}
\begin{equation}\label{23.b}
  \implies -i\hbar\pdv{\psi^\dagger}{t}\psi=-\frac{\hbar c}{i}\sum_k\pdv{\psi^\dagger}{x^k }\a_k\psi+m_0c^2\psi^\dagger \b\psi
\end{equation}
Restando \eqref{23.b} de \eqref{23.a} tenemos
\begin{equation}
  i\hbar\left(\psi^\dagger\pdv{\psi}{t}+\pdv{\psi^\dagger}{t}\psi\right)=\frac{\hbar c}{i}\left(\sum_k\psi^\dagger \a_k\pdv{\psi}{x^k}+\pdv{\psi^\dagger}{x^k}\a_k\psi\right)
\end{equation}
\begin{equation}
  i\hbar\pdv{t}(\psi^\dagger\psi)=\frac{\hbar c}{i}\sum_k\pdv{x^k}(\psi^\dagger\a_k\psi)
\end{equation}
\begin{equation}
   i\hbar\pdv{t}(\psi^\dagger\psi)=-i\hbar\sum_k\pdv{x^k}(c\psi^\dagger\a_k\psi)=0
\end{equation}
Podemos escribir
\begin{equation}
  \pdv{\rho}{t}+\nabla\cdot\vec{J}=0
\end{equation}
donde
\begin{equation}
  \rho=\psi^\dagger\psi,\qquad \vec{J}=c\psi^\dagger\vec{\a }\psi
\end{equation}
Integrando en el espacio,
\begin{equation}
  \int_V\pdv{\rho}{t}\dd^3x+\int_V\nabla\cdot\vec{J}\dd^3x=0
\end{equation}
\begin{equation}
  \pdv{t}\int_V\rho\dd^3x=-\int_V\nabla\cdot\vec{J}\dd^3x=-\int_V\vec{J}\cdot\dd\vec{S}=0
\end{equation}
\begin{equation}
  \implies \pdv{t}\int_V\rho\dd^3x=0
\end{equation}
Es decir, $\rho$ se puede interpretar como una densidad de probabilidad definida positiva.
\begin{equation}
  \rho=\psi^\dagger\psi=\mqty(\psi_1^*&\psi_2^*&\cdots \psi_N^*)\mqty(\psi_1\\\psi_2\\\vdots\\\psi_N)
\end{equation}
\begin{equation}
  \rho=|\psi_1|^2+|\psi_2|^2+\cdot +|\psi_N|^2>0
\end{equation}

\subsection{Ecuación de Dirac en notación $4$-dimensional}
Consideremos la ecuación de Dirac
\begin{equation}
  i\hbar\pdv{\psi}{t}=\frac{\hbar c}{i}\left(\a_1\pdv{x^1}+\a_2\pdv{x^2}+\a_3\pdv{x^3}\right)\psi+\b m_0c^2\psi
\end{equation}
\begin{equation}
  i\hbar\pdv{\psi}{t}=-i\hbar \left(c\a_1\pdv{x^1}+c\a_2\pdv{x^2}+c\a_3\pdv{x^3}\right)\psi+\b m_0c^2\psi\qquad /\frac{\b }{c}
\end{equation}
\begin{equation}
  i\hbar\b\pdv{\psi}{(ct)}=-i\hbar\left(\b\a_1\pdv{x^1}+\b\a_2\pdv{x^2}+\b\a_3\pdv{x^3}\right)\psi+\b^2m_0c\psi
\end{equation}
\begin{equation}
  i\hbar\left[\b\pdv{x^0}+\b\a_1\pdv{x^1}+\b\a_2\pdv{x^2}+\b\a_3\pdv{x^3}-m_0c\psi\right]=0
\end{equation}
Definimos $\gamma^0=\b $ y $\gamma^{i}=\b\a^{i}$. Luego, lo anterior queda
\begin{equation}
  i\hbar\left[\gamma^0\pdv{x^0}+\gamma^1\pdv{x^1}+\gamma^2\pdv{x^2}+\gamma^3\pdv{x^3}\right]\psi-m_0c\psi=0
\end{equation}
\begin{equation}
 \boxed{ i\hbar\gamma^m \partial_\m \psi-m_0c\psi=0}
\end{equation}
conocida como la \textit{ecuación de Dirac} en notación relativista. Usando la notación de Feymann
\begin{align}
  \slashed{A}=\gamma^\m A_\m \implies \gamma^\m \partial_\m =\slashed{\partial}
\end{align}
\begin{equation}
  i\hbar\slashed{\partial}\psi-m_0c\psi=0
\end{equation}
\begin{equation}
  \left(i\hbar\slashed{\partial}-m_0c\right)\psi=0
\end{equation}
En notación de coordenadas naturales $\hbar=1,c=1$, se tiene
\begin{equation}
  (i\slashed{\partial}-m_0)\psi=0
\end{equation}
\begin{equation}
  (i\gamma^\m \partial_\m -m_0)\psi=0
\end{equation}
Si $m=m_0$,
\begin{equation}\label{23.dirac}
\boxed{  (i\gamma^\m \partial_\m -m)\psi=0}
\end{equation}
La ecuación de Dirac \eqref{23.dirac} puede ser obtenida a partir del llamado \textit{Lagrangeano de Dirac}
\begin{equation}\label{23.LagDirac}
  \mathcal{L} _0=\bar{\psi}(i\gamma^\m \partial_\m -m)\psi
\end{equation}
donde $\bar{\psi}=\psi^\dagger \gamma^0$.

\begin{obs}
	Los anticonmutadores de $\a_i$, $\b$ en notación cuadri-dimensional dan lugar al álgebra de Clifford. En efecto, primero
	\begin{equation}
  \gamma^0\gamma^0+\gamma^0\gamma^0 =\b^2+\b^2=2=2\d_{0}
\end{equation}
\begin{equation}\label{23.1}
\boxed{   \gamma^0\gamma^0+\gamma^0\gamma^0=2\d_{00}=2g^{00}}
\end{equation}
Segundo
\begin{equation}
  \gamma^{i}\gamma^0+\gamma^0\gamma^{i} =\b\a_i\b +\b \b \a_i=\a_i\b^2+\b^2\a_i=-\a_i+\a_i=0
\end{equation}
\begin{equation}\label{23.2}
\boxed{  \gamma^{i}\gamma^0+\gamma^0\gamma^{i}=0}
\end{equation}
Y tercero
\begin{align}
  \gamma^{i}\gamma^j+\gamma^j\gamma^{i}&=\b\a_i\b\a_j+\b\a_j\b\a_i\\
  &=-\a_i\b^2\a_j-\a_j\b^2\a_i\\
  &=-(\a_i\a_j+\a_j\a_i)\\
  &=-2\d_{ij}
\end{align}
\begin{equation}\label{23.3}
\boxed{  \gamma^{i}\gamma^j+\gamma^j\gamma^{i}=-2\d_{ij}=-2g^{ij}}
\end{equation}
De \eqref{23.1}, \eqref{23.2} y \eqref{23.3} podemos escribir
\begin{equation}
\boxed{  \gamma^\m \gamma^\n +\gamma^\n \gamma^\m =2g^{\m\n }\equiv 2\eta^{\m\n }}
\end{equation}
la cual corresponde al \textit{álgebra de Clifford.}
\end{obs}

\subsection{Electrodinámica cuántica a partir del principio de gauge}
Consideremos el Lagrangeano de Dirac \eqref{23.LagDirac}
\begin{equation}
  \mathcal{L} _0=\bar{\psi}(x)\left(i\gamma^\m \partial_\m -m\right)\psi(x)
\end{equation}
Bajo una transformación $U(1)$ global, los campos $\psi$ transforman como
\begin{equation}
  \psi(x)\to \psi'(x)=e^{i\a }\psi(x),\qquad \a\in\mathbb{R}
\end{equation}
¿Cómo transforma $\bar{\psi}(x)$? Sabemos que $\bar{\psi}(x)=\psi^\dagger(x)\gamma^0 $, lo que implica que
\begin{align}
  \bar{\psi}'(x)=\psi'^\dagger(x)\gamma^0 &=(\psi')^\dagger\gamma^0 \\
  &=(e^{i\a })^\dagger \gamma^0 \\
  &=\psi^\dagger e^{-i\a }\gamma^0 \\
  &=\psi^\dagger \gamma^0 e^{-i\a }\\
  &=\bar{\psi}e^{-i\a }
\end{align}
Así, para el caso de $U(1)$ global, se tiene \footnote{Para el caso de $U(1)$ local se debe tener cuidado con la conmutación entre los objetos que dependen de las coordenadas.}
\begin{equation}
  \psi\to \psi'=e^{i\a }\psi
\end{equation}
\begin{equation}
  \bar{\psi}\to \bar{\psi}'=\bar{\psi}e^{-i\a }
\end{equation}

\begin{teor}
	El Lagrangeano de Dirac es unvariante bajo $U(1)$ global.
\end{teor}
\begin{prueba}
	\begin{equation}
  \L'_0 =\bar{\psi}'i\gamma^\m\partial_\m \psi'-m\bar{\psi}'\psi'
\end{equation}
Veamos como queda el segundo término:
\begin{equation}
  m\bar{\psi}'\psi'=m\bar{\psi}e^{-i\a }e^{i\a }\psi=m\bar{\psi}\psi \quad \checkmark
\end{equation}
Ahora el primero:
\begin{align}
  i\bar{\psi}'\gamma^\m \partial_\m \psi'&=i\bar{\psi}e^{-i\a }\gamma^\m \partial_\m (e^{i\a }\psi),\quad \a\in\mathbb{R}\\
  &=i\bar{\psi}e^{-i\a }e^{i\a }\gamma^\m \partial_m \psi\\
  &=i\bar{\psi}\gamma^\m \partial_\m \psi\qquad \checkmark \qed 
\end{align}
Notemos que para el caso de $U(1)$ local, esto último ya no se cumple, debido a que $\a$ sería una función de las coordenadas.
\end{prueba}

¿Cuál es la razón de la invariancia? La razón es que la derivada de un campo es también un campo, por lo cual debe transformar como un campo, es decir,
\begin{equation}
  (\partial_\m \psi)'=e^{i\a }(\partial_\m \psi)
\end{equation}
\begin{align}
  \partial'_\m \psi'&=e^{i\a }(\partial_\m \psi)\\
  \partial'_\m e^{i\a }\psi&=e^{i\a }\partial_\m \psi\\
  \partial'_\m e^{i\a }&=e^{i\a }\partial_\m \\
\implies \Aboxed{ \partial'_\m &=e^{i\a }\partial_\m e^{-i\a }}
\end{align}
Es por eso que en el caso de $U(1)$ global $\partial_\m =\partial'_\m $, pero para el caso local esto ya no es cierto.

La idea fundamental de Weyl fue pasar de transformaciones globales a locales, es decir, pasar de $U(1)_{\rm global}$  a $U(1)_{\rm local}$. 

Para el caso de $U(1)$ local, se tiene
\begin{equation}
  \psi(x)\to \psi'(x)=e^{i\a (x)}\psi(x)
\end{equation}
\begin{equation}
  \bar{\psi}(x)\to \bar{\psi}'=\bar{\psi}(x)e^{-i\a (x)}
\end{equation}
Consideremos de nuevo el Lagrangeano de Dirac \eqref{23.LagDirac},
\begin{align}
  \L'_0&=\bar{\psi}'(i\gamma^\m \partial_\m -m)\psi'\\
  &=i\bar{\psi}'\gamma^\m \partial_\m \psi'-m\bar{\psi}'\psi'
\end{align}
Con el segundo término hay problema, como fue mostrado anteriormente. Sin embargo, con el primero ya no
\begin{align}
  i\bar{\psi}'\gamma^\m \partial_\m \psi'&=i\bar{\psi}e^{-i\a (x)}\gamma^\m \partial_\m (e^{i\a (x)}\psi)\\
  &=i\bar{\psi}e^{-i\a (x)}\gamma^\m \left[e^{i\a (x)}\partial_\m \psi+i(\partial_\m \a(x))e^{i\a (x)}\psi\right]\\
  &=i\bar{\psi}e^{-i\a (x)}\gamma^\m e^{i\a (x)}\left[\partial_m\psi +i (\partial_\m \a(x))\psi\right]\\
  &=i\bar{\psi}\gamma^\m \left[\partial_\m +i\partial_\m \a (x)\right]\psi
\end{align}
Lo que implica que
\begin{align}
  \L'_0 &=i\bar{\psi}\gamma^\m \partial_\m \psi-\bar{\psi}\gamma^\m \psi\partial_\m \a(x)-m\bar{\psi}\psi\\
  &=\bar{\psi}(i\gamma^\m \partial_\m -m)\psi-\bar{\psi}\gamma^\m \psi\partial_\m \a(x)
\end{align}



















































