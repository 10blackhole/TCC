\section{Clase 24}
Habíamos visto que:
\begin{enumerate}
	\item El Lagrangeano $\L_0=\bar{\psi}(i\g^\m \partial_\m -m)\psi\implies \L'=i\psi'\g^\m (\partial_\m \psi)'-m\bar{\psi}\psi$ es invariante bajo $U(1)$ global, debido a que $\partial_\m \psi$ transforma como $\psi$
	\begin{equation}
  \psi'=e^{i\a }\psi\leftrightarrow (\partial_\m \psi)'=e^{i\a }(\partial_\m \psi)
\end{equation}

\item 	
El Lagrangeano $\L_0=\bar{\psi}i\g^\m \partial_\m \psi-m\bar{\psi}\psi$ \textit{no} es invariante bajo $U(1)$ local
\begin{equation}
  \psi'(x)=e^{i\a(x)}\psi(x)
\end{equation}
\begin{equation}
  \bar{\psi}'(x)=\bar{\psi}(x)e^{-i\a(x)}
\end{equation}
En efecto,
$\L'=i\bar{\psi}'\g^\m (\partial_\m \psi)'-m\bar{\psi}'\psi'$ no es invariante bejo $U(1)$ local debido a que $(\partial_\m \psi)'$ no transforma como $\psi'$
\begin{equation}
  (\partial_\m \psi)'=e^{i\a(x)}(\partial_\m +i\partial_\m \a(x))\psi(x)
\end{equation}
\end{enumerate}

La solución al problema es introducir una derivada tal que la derivada del campo transforme como el campo. Sea $D_\m $ dicha derivada:
\begin{equation}\label{24.1}
  (D_\m \psi)'=e^{i\a(x)}D_\m \psi
\end{equation}
la cual obliga a escribir el Lagrangeano como
\begin{align}
  \L_0&=\bar{\psi}(i\g^\m D_\m -m)\psi\\
  &=i\bar{\psi}\g^\m D_\m \psi-m\bar{\psi}\psi
\end{align}
\begin{equation}
  \implies \L'=\bar{\psi}'\g^\m (D_\m\psi)'-m\bar{\psi}'\psi'=\L_0
\end{equation}

¿Cómo definimos la derivada $D_\m $? La respuesta es que dado que
\begin{enumerate}
	\item $\partial_\m '=e^{i\a(x)}(\partial_\m +i\partial_\m \a(x))e^{-i\a(x)}$
	\item De \eqref{24.1} $D_\m '=e^{i\a(x)}D_\m e^{-i\a(x)}$
\end{enumerate}
podemos postular
\begin{equation}
  \boxed{D_\m =\partial_\m +iA_\m }
\end{equation}
es decir, para recuperar la invariancia, introducimos un nuevo campo $A_\m$ llamado \textit{campo de gauge, potencial de gauge, o campo compensante}, cuya ley de transformación debe ser encontrada.

\begin{teor}
	Bajo una transformación $U(1)$ local, el nuevo campo potencial de gauge transforma como
	\begin{equation}
  A_\m '=A_\m -\partial_\m \a(x)
\end{equation}
\end{teor}

\begin{prueba}
	Sabemos que
	\begin{equation}
  \partial_\m =e^{i\a(x)}(\partial_\m +i\partial_\m \a(x))e^{-i\a(x)}
\end{equation}
\begin{equation}
  D_\m '=e^{i\a(x)}D_\m e^{-i\a(x)}
\end{equation}
\begin{equation}
  \implies \partial_\m '+iA_\m '=e^{i\a(x)}(\partial_\m +iA_\m )e^{-i\a(x)}
\end{equation}
\begin{equation}
  e^{i\a(x)}(\partial_\m +i\partial_\m \a(x))e^{-i\a(x)}+iA_\m '=e^{i\a(x)}(\partial_\m +iA_\m )e^{-i\a(x)}
\end{equation}
\begin{equation}
  \cancel{e^{i\a(x)}\partial_\m e^{-i\a(x)}}+e^{i\a(x)}[i\partial_\m \a(x)]e^{-i\a(x)}+iA_\m '=\cancel{e^{i\a(x)}\partial_\m e^{-i\a(x)}}+e^{i\a(x)}(iA_\m )e^{-i\a(x)}
\end{equation}
\begin{equation}
  i\partial_\m \a(x)+iA_\m '=iA_\m 
\end{equation}
\begin{equation}
  \implies \boxed{A_\m '=A_\m -\partial_\m \a(x)}\qquad \Box 
\end{equation}
\end{prueba}
Así entonces, el Lagrangeando $\L_0$ invariante bajo $U(1)$ local y bajo $A_\m '=A_\m -\partial_\m \a(x)$ es
\begin{equation}
  \L =i\bar{\psi}\g^\m D_\m \psi-m\bar{\psi}\psi,\qquad D_\m =\partial_\m +iA_\m 
\end{equation}
\begin{align}
  \L =&i\bar{\psi}\g ^\m (\partial_\m +iA_\m )\psi-m\bar{\psi}\psi\\
  &=i\bar{\psi}\g^\m \partial_\m \psi-\bar{\psi}\g^\m A_\m \psi-m\bar{\psi}\psi
\end{align}
\begin{equation}
\boxed{  \L =\bar{\psi}(i\g^\m \psi_\m -m)\psi-\bar{\psi}\g^\m \psi A_\m }
\end{equation}
Este Lagrangenao muestra la interacción del campo compensante $A_\m $ con la materia $\psi$. Sin embargo, no aparece el término cinético correspondiente al campo $A_\m $. Para obtener dicho término seguimos el procedimiento introducido en la teoría general de la Relatividad,
\begin{equation}
  [D_\m ,D_\n ]\psi=iF_{\m\n }
\end{equation}
Calculemos el conmutador:
\begin{align*}
  [D_\m ,D_\n ]\psi&=D_\m D_\m \psi-D_\n D_\m \psi\\
  &=(\partial_\m +iA_\m )(\partial_\n  +iA_\n  )\psi -(\partial_\n  +iA_\n  )(\partial_\m +iA_\m )\psi\\
  &=\cancel{\partial_\m \partial\n \psi}+i\partial_\m (A_\n \psi)+iA_\m \partial_\n \psi-\cancel{A_\m A_\n \psi}-\cancel{\partial_\n \partial_\m \psi}-i\partial_\n (A_\m \psi)-iA_\n \partial_\m \psi+\cancel{A_\n A_\m \psi}\\
  &=i(\partial_\m A_\n )\psi+iA_\m \psi_\n \psi+iA_\m \partial_\n \psi-i(\partial_\n A_\m )\psi-iA_\n \partial_\m \psi-iA_n\partial_\m \psi\\
  &=i(\partial_\m A_\n -\partial_\n A_\mathcal)\psi\\
  &=iF_{\m\n }\psi
\end{align*}
\begin{equation}
  \implies \boxed{F_{\m\n}=\partial_\m A_\n -\partial_\n A_\m }
\end{equation}
El Lagrangeano que corresponde al campo $A_\m $ es
\begin{equation}
  \L_F=-\frac{1}{4}F_{\m\n}F^{\m\n }
\end{equation}
Así entonces, tenemos que el Lagrangeano invariante bajo las transformaciones
\begin{equation}
  \psi'=e^{i\a (x)}\psi,\qquad \bar{\psi}'=\bar{\psi}e^{-i\a (x)},\qquad A_\m'=A_\m -\partial_\m \a(x)
\end{equation}
es dado por
\begin{equation}
 \boxed{ \L_{\rm QED}=-\frac{1}{4}F_{\m\n }F^{\m\n }+\bar{\psi}(i\g^\m \partial_\m -m)\psi-\bar{\psi}\g^\m \psi A_\m }
\end{equation}

\subsection{Teorías de Yang-Milss}
\underline{1954}: Postularon que las fuerzas nucleares fuertes podrían ser estudiadas como la teoría de campos análoga a la teoría de la Electrodinámica. También postularon que del mismo modo que la simetría local $U(1)$ que describe la QED, la simetría de las fuerzas nucleares fuertes era la simetría $SU(2)$.

\underline{1932}: Heisenberg postuló que el protón y el neutrón son dos estados de una misma partícula llamada \textit{nucleón}.

Sabemos que los núcleos atómicos están compuestos de protones y neutrones. Los protones tienen carga eléctrica positiva, es decir, se repelen eléctricamente.

¿Cómo es posible que los núcleo permanezcan estables? La respuesta es que existe una fuerza atractiva entre protones y protones, entre protones y netrones, y entre neutrones y neutrones, de tal intensirdad que impide la desintegración del núcleo debido a la repulsión eléctrica. Esta fuerza es de muy poco alcance $\sim 10^{-15}$ m. A distancias mayores decae violentamente.

¿Cómo identificar a los ojos de la física nuclear protones y neutroes?
Así como el electrón tiene dos esados identificados por el spin: spin $+1/2$ up ($\uparrow$) y spin $-1/2$ down ($\downarrow$), Heisenberg definió un número cuántico para los núcleo atómicos llamado \textit{isospin}. Así, un núcleo con $Z$ protones y $N$ neutrones tiene como tercera comopoentes del isospin a
\begin{equation}
  I_3=\frac{Z-N}{2}
\end{equation}
Por lo tanto, para el protón $I_3=+1/2$, mientras que para el neutrón $I_3=-1/2$.

La diferenca entre las masa del protón y del neutrón se debe sólo a la interacción electromagnética. Luego, a los ojos de la fuerza nuclear ellos son indistinguibles, salvo por la tercera componente del isospin.

Las masas del proón, neutrón y del electrón son
\begin{align}
  m_p&\approx 1.6726\times 10^{-27} \text{kg}\\
  m_N&\approx 1.6749\times 10^{-27} \text{kg}\\
  m_p&\approx 9.1\times 10^{-31} \text{kg}
\end{align}
respectivamete.











































