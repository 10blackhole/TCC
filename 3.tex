\section{Clase 3}
\subsection{Simetrías y leyes de conservación}
La homogeneidad del tiempo nos lleva a la conservación de la energía. Que el tiempo sea homogéneo significa que no hay instantes privilegeados. Los resultados de un experimento nno dependen del instantes en que se lleven a cabo, es decir, si llevamos a cabo un experimento para $t=t$ será tambien el mismo en $t'=t+t_0$. La función de Lagrange (que describe un sistema físico) debe ser invariante bajo un desplazamiento temporal, es decir, $\Lag$ es invariante bajo la transformación $t\to t'=t+t_o$ o $\delta t=t'-t=t_o$. Esto se cumplirá sólo si $\Lag$ no depende expl{icitaente del tiempo, es decir,
\begin{equation}
  \pdv{L}{t}=0
\end{equation}
Así,
\begin{equation}\label{3.1}
  \dv{L}{t}=\pdv{L}{q_i}\dot{q}_i+\pdv{L}{\dot{q}_i}\ddot{q}_i
\end{equation}
De la derivada de Euler-Lagrange, sabemos
\begin{equation}
  [L]_i=\pdv{L}{q_i}-\dv{t}\pdv{L}{\dot{q}_i}\qquad \Rightarrow\qquad \pdv{L}{q_i}=[L]_i+\dv{t}\pdv{L}{\dot{q}_i}
\end{equation}
Reemplazando en \eqref{3.1}
\begin{align}
  \dv{L}{t}&=\left([L]_i+\dv{t}\pdv{L}{\dot{q}_i}\right)\dot{q}_i+\pdv{L}{\dot{q}_i}\ddot{q}_i\\
  &=[L]_i\dot{q}_i+\dv{t}\pdv{L}{\dot{q}_i}\dot{q}_i+\pdv{L}{\dot{q}_i}\ddot{q}_i
\end{align}
de donde se obtiene
\begin{align}
  [L]_i\dot{q}_i&=\dv{L}{t}-\dv{t}\dot{q}_i\pdv{L}{\dot{q}_i}\\
  &=-\dv{t}\left(\dot{q}_i\pdv{L}{\dot{q}_i}-L\right)
\end{align}
Para trayectorias on-shell, es decir, para el espacio de soluciones de la ecuación de Euler-Lagrange $[L]_i=0$, se tiene
\begin{equation}
  \dv{t}\left(\dot{q}_i\pdv{L}{\dot{q}_i}-L\right)=0
\end{equation}
Pero sabemos que la función de Hamilton es dada por
\begin{equation}
  H=\dot{q}_ip_i-L=E
\end{equation}
Luego,
\begin{equation}
  \dv{H}{t}=\dv{E}{t}=0
\end{equation}
es decir, \textbf{la homogeneidad del tiempo implica la conservación de la energía.}

Por otra parte, la homogeneidad del espacio conduce a la cnservación del momentum lineal. Que el espacio sea homogeneo nos dice que todos los puntos son equivalentes y no hay posiciones privilegiadas en el espacio. Esto implica que la función de Lagrange $\Lag$ debe ser invariante bajo una traslación espacial de la forma
\begin{equation}
  q_i\to q'_i=q_i+a_i\qquad \text{ó}\qquad \delta q_i=q'_i-q_i=a_i
\end{equation}
Así 
\begin{equation}
  \delta_q L=\pdv{L}{q_i}\delta q_i=0
\end{equation}
De la derivada de Euler-Lagrange,
\begin{equation}
  [L]_i=\EL 
\end{equation}
\begin{equation}
  \implies \d_qL=\left([L]_i+\dv{t}\pdv{L}{\qd_i}\right)a_i=0
\end{equation}
\begin{equation}
  \implies [L]_ia_i+a_i\dv{t}p_i=0
\end{equation}
Para trayectorias on-shell, es decir, cuando las ecuaciones de Euler-Lagrange se cumplen, se tiene
\begin{equation}
  \dv{p_i}{t}=0\quad \implies \quad p_i=\text{constante del movimiento}
\end{equation}
es decir, \textbf{la homogeneidad del espacio implica la conservación del momentum.}

Finalmente, se puede mostrar que la \textbf{isotropía del espacio implica la conservación del momentum angular.} Que el espacio sea isótropo quiere decir que todas las direcciones son privilegiadas. Luego, la función de Lagrange es invariante bajo rotaciones espaciales.












