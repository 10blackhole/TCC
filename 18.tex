\section{Clase 18}
Teníamos que
\begin{equation}
  \d S=\epsilon\int_\Omega\dd^4x\left[[\L ]_\a (\eta^\a -\partial_\m \psi^\a \xi^\m )+\partial_\m \left(\chi^\a \pdv{\L }{(\partial_\m \partial^\a )}+\L \xi^\m  \right)\right]
\end{equation}
De aquí vemos:
\begin{teor}\label{teo1}
	Sea $\xi_r^\m =\xi_r^\m (\psi,\partial \psi,x)$, $\eta_r^\a  =\eta_r^\a  (\psi,\partial \psi,x)$ funciones continuas. Si $\epsilon^r$, $r=1,...,N$ son parámetros constantes tales que
	\begin{align}
  \d x^\m &=\epsilon^r\xi_r^\m (\psi,\partial\psi ,x)\\
  \d \psi^\a  &=\epsilon^r\eta_r^\m (\psi,\partial\psi ,x)
\end{align}
entonces existen las siguientes $N$ identidades
\begin{equation}
  (\eta_r^\a -\partial_\m \psi^\a \xi_r^\m )[\L ]_a =\nabla_\m \left(\chi_r^\a \pdv{\L }{(\partial_\m \psi^\a )}+\L (x)\xi_r^\m \right)
\end{equation}
por lo cual
\begin{equation}
  J_\m =\chi_r^\a \pdv{\L }{(\partial_\m  \psi^\a )}+\L \xi_r^\m 
\end{equation}
es una cantidad conservada on-shell.
\end{teor}

\begin{teor}\label{teo1}
	Sean $\hat{\xi}_r^\m $, $\hat{\eta}_r^\a $ operadores. Si $\epsilon^r, r=1,..,N$ son funciones continuas, tales que
	\begin{align}
  \d x^\m &=\epsilon^r\hat{\xi}_r^\m \\
  \d\psi^\a &=\epsilon^r\hat{\eta}_r^\a 
\end{align}
entonces existen $N$ identidades de la forma
\begin{equation}
  \hat{\eta}_r^{*\a}[\L ]_\a -\hat{\xi}_r^{*\m }(\partial_\m \psi^\a [\L ]_\a )=0
\end{equation}
donde $*$ indica el adjunto del operador.
\end{teor}

\subsection{Corrientes y cargas}
En el Teorema \ref{teo1} se establece que
	\begin{align}
  \d x^\m &=\epsilon^r\xi_r^\m (\psi,\partial\psi ,x)\\
  \d \psi^\a  &=\epsilon^r\eta_r^\m (\psi,\partial\psi ,x)
\end{align}
donde $\epsilon^r$ son parámetros constantes. Por simplicidad consideremos las siguientes transformaciones
\begin{equation}
  \d x^\m =\epsilon^rA_m^r(x),\qquad \d \psi^\a =\epsilon^r B_r^\a (x)
\end{equation}
de manera que
\begin{align}
  \db \psi^\a &=\d\psi^\a -\partial\psi^\a \d x^\m\\& =\epsilon^rB_r^\a (x)-\partial_\m\psi^\a \epsilon^r A_r^\m (x)\\
  &=\epsilon^r(B_r^\a -\partial_\m\psi^\a A_r^\m )\\
  &=\epsilon^r\chi_r^\a 
\end{align}
pero
\begin{equation}
  \d S=\int\dd^4x\left[[\L ]_\a \db\psi^\a +\partial_\m \left(\pdv{\L }{(\partial_\m\psi)}\db\psi^\a +\L(x)\d x^\m \right)\right]
\end{equation}
luego,
\begin{align}
  \d \L &=[\L ]_\a \db\psi^\a +\partial_\m \left(\pdv{\L }{(\partial_\m\psi)}\db\psi^\a +\L(x)\d x^\m +\Sigma^\m  \right)\\
  &=[L]_\a \db\psi^\a +\partial_\m J^\m 
\end{align}
donde 
\begin{equation}
  J^\m =\pdv{\L }{(\partial_\m\psi)}\db\psi^\a +\L(x)\d x^\m +\Sigma^\m 
\end{equation}

Definimos el \textit{pseudomomentum} $\pi ^{\a\m }=\pdv{\L }{(\partial_\m \psi^\a )}$, de manera que
\begin{equation}
    J^\m =\pi ^{\a \m} \db\psi^\a +\L\d x^\m +\Sigma^\m 
\end{equation}

Dado que $\db\psi^\a =\d\psi^\a -\partial_\n \psi^\a \d x^\n $, se tiene
\begin{align}
    J^\m &=\pi ^{\a \m} \db\psi^\a -\pdv{\L }{(\partial_\m \psi^\a )}\partial_\n \psi^\a \d x^\n +\L \d x^\m +\Sigma^\m 
\end{align}
\begin{equation}
  \implies \boxed{J^\m =\pi^{\a\m }\d \psi^\a -T^{\m }_{~\n }\d x^\n +\Sigma^\m }
\end{equation}
donde
\begin{align}
  T^{\m }_{~\n }&=\pdv{\L }{(\partial_\m\psi^\a )}\partial_\n \psi^\a -\delta^\m_\n \L\\
  &= \pi^{\a\m }\partial_\n \psi^\a -\d^\m_\n\L 
\end{align}
Dado que $\d\psi^\a =\epsilon^rB_r^\a $, $\d x^\n =\epsilon^r\xi_r\n $ y $\Sigma^m =\epsilon\sigma_r^\m $, se tiene
\begin{align}
  J^\m &=\epsilon^r\pi^{\a\m }B_r^\a -T^\m _{~\n }\epsilon^r\xi_r^\n +\epsilon^r\sigma_r^\m \\
  &=\epsilon^r \left(\pi^{\a\m }B_r^\a -T^\m _{~\n }\xi_r^\m +\sigma_r^\m \right)\\
  &=\epsilon^rJ_r^\m 
\end{align}




Integrando sobre el $3$-espacio la ley de conservación $\partial_\m J^\m =0$, se tiene
\begin{align}
  \int\dd^3x\partial_\m J^\m &=0\\
  \int\dd^3x\partial_0 J^0+\int\dd^3x\partial_iJ^{i}&=0\\
  \implies \int\dd^3x\partial_0 J^0&=-\int\dd^3x\partial_iJ^{i}\\
  &=-\int\dd^2x J^{i}\dd S_i
\end{align}
asumiendo que los campos se anulan en el infinito,
\begin{equation}
  \partial_0\int\dd^3xJ^0=0
\end{equation}
luego, la carga
\begin{equation}
  C=\int\dd^3xJ^0
\end{equation}
se conserva.

\begin{teor}
	La variación $\db$ de los campos $\psi^\a $ pueden ser expresadas en función de las cargas
	\begin{equation}
  \db\psi^\a =[\psi^\a ,\epsilon^rC_r]
\end{equation}
\end{teor}

\begin{teor}
	Las cargas de Noether satisfacen la relación de conmutación
	\begin{equation}
  [C_r,C_s]=f_{rst}C_t+\gamma_{rs}
\end{equation}
\end{teor}

\subsection{Simetrías de Poincaré y leyes de conservación}
\textbf{a) Invariancia bajo traslaciones}: La homogeneidad del espacio-tiempo implica invariancia bajo traslaciones,
\begin{equation}
  x'^\m =x^\m +\epsilon^\m \implies \d x^\m =\epsilon^\m 
\end{equation}

Esta homogeneidad implica que los campos no cambian bajo traslaciones
\begin{equation}
  \psi'(x')=\psi(x)\implies \d\psi^\a (x)=0
\end{equation}
por lo cual, la corriente
\begin{equation}
  J^\m =\pi^{\a\m }\d \psi^\a -T^{\m }_{~\n }\d x^\n
\end{equation}
toma la forma
\begin{align}
  J^\m =-T^{\m }_{~\n }\d x^\n=-T^{\m }_{~\n }\epsilon^r 
\end{align}
\begin{equation}
  \implies \partial_m J^\m =0\implies \boxed{\partial_\m T^\m _{~\n }=0}
\end{equation}
Integrando sobre el volumen espacial
\begin{equation}
  \int\dd^3x\partial_\m T^\m _{~\n }=0
\end{equation}
\begin{equation}
  \implies \int\dd^3x \partial_0T^0 _{~\n }=0
\end{equation}
\begin{equation}
  \implies \dv{x^0}P_\n =0
\end{equation}
donde
\begin{equation}
  \boxed{P_\n =\int\dd^3x T^0 _{~\n }=0}
\end{equation}

\textbf{b) Invariacia bajo rotaciones de Lorentz}: La isotropía del espacio-tiempo implica invariancia bajo rotaciones espaciales y boosts de Lorentz.

Bajo transformaciones de Lorentz
\begin{equation}
  x'^\m =\Lmn x^\n 
\end{equation}
donde $\Lmn$ son elemento del grupo de Lie-Lorentz, lo que implica que basta estudiarlas transformaciones infinitesimales.
Sabemos que bajo $\Lmn=\delta^\m_\n +\omn $,
\begin{equation}
  \d x^\m =\omn x^\n \equiv \omega^{\m\n }x_\n 
\end{equation}
\begin{equation}
  \d\psi_r=\frac{1}{2}\omega^{\m\n }(J_{\m\n })_{rs}\psi_s
\end{equation}

Consideremos estas transformaciones en la corriente
\begin{equation}
  J_\m =\pdv{\L }{(\partial^\m \psi_r)}\d\psi_r-T_{\m\n}\d x^\n 
\end{equation}
con
\begin{equation}
  T_{\m\n }=\pdv{\L }{(\partial^\m \psi_r)}\psi_r-\eta_{\m\n }\L 
\end{equation}
así,
\begin{align}
  J_\m &=\pdv{\L }{(\partial^\m \psi_r)}\frac{1}{2}\omega^{\m\lambda}(J_{\n\lambda})\psi_s-T_{\m\n }\omega^{\n\lambda}x_\lambda
\end{align}
pero
\begin{align}
  T_{\m\n }\omega^{\n\lambda}&=\frac{1}{2}T_{\m\n }\omega^{\n\lambda}x_\lambda +\frac{1}{2}T_{\m\n }\omega^{\n\lambda}x_\lambda\\
  &=\frac{1}{2}T_{\m\n }\omega^{\n\lambda}x_\lambda +\frac{1}{2}T_{\m\lambda }\omega^{\lambda\n }x_\n \\
  &=\frac{1}{2}\omega^{\n\lambda}(T_{\m\n }x_\lambda-T_{\m\lambda}x_\n )
\end{align}
luego,
\begin{align}
  J_\m &=\pdv{\L }{(\partial^\m \psi_r)}\frac{1}{2}\omega^{\n\lambda}(J_{\n\lambda})_{rs}\psi_r+\frac{1}{2}\omega^{\n\lambda}(T_{\m\n }x_\lambda-T_{\m\lambda}x_\n )\\
  &=\frac{1}{2}\omega^{\n\lambda}\left(x_\n T_{\m\lambda}-x_\lambda T_{\m\n }+\pdv{\L }{(\partial^\m \psi_r)}\left(J_{\n\lambda}\right)_{rs}\psi_s\right)\\
  &=\frac{1}{2}\omega^{\n\lambda}M_{\m\n\lambda}
\end{align}
donde
\begin{equation}
 \boxed{ M_{\m\n\lambda}=x_\n T_{\m\lambda}-x_\lambda T_{\m\n }+\pdv{\L }{(\partial^\m \psi_r)}\left(J_{\n\lambda}\right)_{rs}\psi_s}
\end{equation}
%conocido como el \textit{tensor corriente}.

Así,
\begin{align}
  C=\int\dd^3x J_0=\frac{1}{2}\omega^{\n\lambda}\int\dd^3xM_{0\n\lambda}=\frac{1}{2}\omega^{\n\lambda}M_{\n\lambda}
\end{align}
donde
\begin{equation}
  M_{\n\lambda }=\int\dd^3xM_{0\n\lambda}
\end{equation}
luego,
\begin{equation}
  M_{\n\lambda}=\int\dd^3x\left(x_\n T_{0\lambda}-x_\lambda T_{0\n }+\pdv{\L }{(\partial^0\psi_r)}\left(J_{\n\lambda}\right)_{rs}\psi_s\right)
\end{equation}
Consideremos solo la parte espacial $\n =n,\lambda=l$
\begin{equation}
  M_{nl}=\int\dd^3x\left(x_\n T_{0l}-x_l T_{0n }+\pdv{\L }{(\partial^0\psi_r)}\left(J_{nl}\right)_{rs}\psi_s\right)
\end{equation}
Escribiendo
\begin{equation}
  M_{nl}=L_{nl}+S_{nl}
\end{equation}
donde
\begin{align}
  L_{nl}&=\int\dd^3x(x_nT_{0l}-x_lT_{0n})\\
  S_{nl}&=\int\dd^3x\pdv{\L }{(\partial^0 \psi_r)}\left(J_{nl}\right)_{rs}\psi_s
\end{align}

Dado que
\begin{equation}
  T_{\m\n }=\pdv{\L }{(\partial^\m \psi_r)}\partial_\n \psi_r-\eta_{\m\n }\L 
\end{equation}
\begin{equation}
 \implies T_{0l}=\pdv{\L }{(\partial^0 \psi_r)}\partial_l\psi_r-\cancelto{0}{\eta_{0l}\L }
\end{equation}
\begin{equation}
  \implies L_{nl}=\int\dd^3x\left(x_n\pdv{\L }{(\partial^0 \psi_r)}\partial_l\psi_r-x_l\pdv{\L }{(\psi^0 \psi_r)}\partial_n\psi_r\right)
\end{equation}
donde se obtiene el \textit{momentum orbital} dado por
\begin{equation}
  L_{nl}=\int\dd^3x\pdv{\L }{(\partial^0\psi_r)}\left(x_n\partial_l-x_l\partial_n\right)\psi_r
\end{equation}































































































